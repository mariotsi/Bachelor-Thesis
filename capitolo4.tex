%!TEX root = Tesi__Simone_Mariotti.tex
\chapter{Implementazione}
\fancyhead[R]{\bfseries Implementazione} 	
\fancyfoot[C]{\thepage }
\section {Android}
L'app per Android è stata sviluppata avendo come priorità la modularità. E' 
infatti molto semplice cambiare totalmente il comportamento del robot o la 
codifica dei messaggi sostituendo o modificando una singola classe senza 
coinvolgere il resto del codice.\\
La modularizzazione a grana più grossa è a livello di package\footnote{Un contenitore 
che racchiude classi che svolgono compitini affini}: ci sono tre package,
ognuno con un compito preciso, e sono \textbf{logic}, \textbf{messaging} e \textbf{opencv}.
\subsection {Il package \textit{logic}}
Si occupa di reperire le informazioni dal mondo reale e analizzarle al fine di
eseguire l'azione più adatta.
Contiene tre classi: RobotActivity, RobotLogic e UpdateDirections.
\subsubsection{La classe \emph{RobotActivity}}
Come suggerisce il nome è l'Activity vera e propria, cioè quella classe che il 
sistema operativo istanzia all'avvio dell'app. Essa stessa istanzia e prepara tutti gli 
altri oggetti per l'esecuzione. Implementa due interfacce: \textit{View.OnTouchListener} 
e \textit{CvCameraViewListener}. 
La prima permette di gestire gli input da touch screen senza ricorrere ad una classe esterna,
la seconda è un'interfaccia presente nella libreria OpenCV e permette di ``intercettare''
i frame proveniente dalla camera prima che vengano renderizzati a schermo tramite 
l'override\footnote{Tecnica che permette di ridefinire il comportamento di un metodo 
ereditato} del metodo \textit{OnCameraFrame()}. Ogni frame verrà elaborato e solo alla
fine visualizzato a schermo.
All'avvio si occupa di inizializzare OpenCV tramite la 
callback \textit{BaseLoaderCallback} e ottiene il riferimento all'istanza dell'ADK
tramite l'ADKToolkit. Normalmente sarebbe buona norma interrompere le connessioni 
quando l'app si prepara allo stato di ``pause'' e cioè nel metodo \emph{onPause()}
 purtroppo però un bug dell'ADK non permette di ripristinare la connessione una 
 volta chiusa senza riavviare l'app. Per questa volta quindi andremo contro la 
 ``best practice'' e lasceremo la connessione sempre attiva ma non mancheremo di 
 causare un ``graceful shutdown'' al modulo che gestisce la comunicazione in modo 
 che non consumi risorse inutilmente.

 \subsubsection{La classe \emph{UpdateDirections}}
 Questa classe è un ``singleton''\footnote{E' un design pattern descritto dalla 
 cosiddetta ``Gang of four'' nel libro ``Design Patterns''. 
 Permette la creazione di una sola istanza della classe e ne regola l'accesso. } 
 e si occupa di mostrare all'utente tramite  immagini e testi quello che  il 
 robot sta facendo o quale sarà la sua prossima  mossa. Dovendo agire 
 sull'UI\footnote{User Interface, interfaccia utente} deve essere eseguita sul thread
 che si occupa dell'UI, per questo implementa l'interfaccia \emph{Runnable} e 
 ogni sua esecuzione è lanciata  tramite \emph{runOnUiThread()}. 
 Le informazioni visualizzabili sono limitate e ben distinte,
 ad ognuna corrisponde un metodo da invocare per visualizzare quella data informazione
 a schermo. I metodi disponibili sono: 
 \begin{itemize}
 \item \emph{left()}: indica che l'obiettivo è visibile è stato individuato e si trova alla sinistra del robot.
 \item \emph{right()}: indica che l'obiettivo è visibile è stato individuato e si trova alla destra del robot.
 \item \emph{aimed()}: indica che l'obiettivo è visibile è stato individuato e si trova esattamente di fronte al robot che si muoverà in quella direzione.
 \item \emph{search()}: indica che l'obiettivo non è visibile, il robot si muoverà secondo l'algoritmo di ricerca.
 \item \emph{found()}: indica che l'obiettivo è visibile e il robot si trova a meno di 30 centrimetri.
 \item \emph{avoidingLeft()}: indica che è presente un ostacolo sul percorso del robot a meno di 30 centimetri. Il robot lo aggirerà verso sinistra. 
 \item \emph{avoidingRight()}: indica che è presente un ostacolo sul percorso del robot a meno di 30 centimetri. Il robot lo aggirerà verso destra.
 \item \emph{chooseColor()}: indica che il robot è in attesa che venga impostato il colore da cercare. 
 \end{itemize}
 La classe espone anche altri metodi che mostrano (\emph{show()}) o nascondono 
 (\emph{hide()}) la porzione di interfaccia che visualizza le indicazioni, oppure bloccano (\emph{lock()})
 e sbloccano (\emph{unlock()}) la possibilità di cambiare le indicazioni visualizzate.

\subsubsection{La classe \emph{RobotLogic}}
\section {Arduino} n