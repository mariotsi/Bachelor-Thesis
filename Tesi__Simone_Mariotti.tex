\documentclass[a4paper,12pt]{report}
%%%%%%%%%%%%%%%%%%%%%%%%%%%%%%%%%%%%%%%%%%%%%%%%%%%%%%%%%%%%%%%%%%%%%%%%%
%%%%%% Definizioni:
\def\titolotesi{Implementazione di un sistema mobile ed autonomo per la ricerca di oggetti in base al colore} % INSERIRE TIOLO DELLA TESI
\def\laureando{Simone Mariotti}       % INSERTIRE NOME COGNOME LAUREANDO
\def\annoaccademico{2013-2014}    % INSERIRE ANNO ACCADEMICO
\def\dedica{TODO: DEDICA}      % INSERIRE DEDICA
\title{\begin{large}\textbf{\titolotesi}\end{large}}
\author{\laureando}
%%%%%% File previsti in input:
%% introduzione.tex    (deve contenere solo il testo, senza \chapter{})
%% capitolo1.tex       (deve iniziare con \chapter{titolo capitolo})
%% capitolo2.tex                           "
%% capitolo3.tex                           "
%% capitolo4.tex                           "
%% conclusioni.tex     (deve contenere solo il testo, senza \chapter{})
%% appendice.tex       (deve contenere solo il testo, senza \chapter{})
%%%%%%%%%%%%%%%%%%%%%%%%%%%%%%%%%%%%%%%%%%%%%%%%%%%%%%%%%%%%%%%%%%%%%%%%%

% Title Page

\usepackage[utf8]{inputenc}
\usepackage[italian]{babel}
\usepackage{fancyhdr}
\usepackage{float}
\usepackage{listings}
\usepackage{color}
\usepackage{xcolor}
\usepackage{textcomp}
\usepackage{gensymb}
\usepackage{amssymb}

\definecolor{dkgreen}{RGB}{0,51,0}
\definecolor{gray}{rgb}{0.5,0.5,0.5}
\definecolor{mauve}{rgb}{0.58,0,0.82}
\renewcommand{\lstlistingname}{Codice}

\definecolor{Maroon}{RGB}{127,2,2}

\lstdefinelanguage{XML}
{
frame=tb, 
showstringspaces=false,
  basicstyle=\ttfamily\footnotesize,
  morestring=[b]",
  moredelim=[s][\bfseries\color{Maroon}]{<}{\ },
  moredelim=[s][\bfseries\color{Maroon}]{</}{>},
  moredelim=[l][\bfseries\color{Maroon}]{/>},
  moredelim=[l][\bfseries\color{Maroon}]{>},
  morecomment=[s]{<?}{?>},
  morecomment=[s]{<!--}{-->},
  commentstyle=\color{dkgreen},
  stringstyle=\color{blue},
  identifierstyle=\color{red},  
frame=tb
}

%\usepackage[hidelinks]{hyperref}
\usepackage{hyperref}

\usepackage{epsfig}
\fancyhf{}	 

\addto\captionsitalian{%
  \renewcommand{\listfigurename}{Elenco delle immagini}%  
}
\renewcommand{\labelitemi}{\scriptsize$\blacksquare$} 
\setlength{\headheight}{15pt}
\pagenumbering{Roman}
%\usepackage{setspace}
\newlength\sinistra
\newlength\corpo
\newlength\pagina
\setlength {\pagina} {21cm}
\setlength {\sinistra} {1.46cm}
\setlength {\corpo} {13.5cm}
\textwidth \the\corpo
\hoffset \the\sinistra
\paperwidth \the\pagina
\linespread{1.6}


\begin{document}
\begin{titlepage}
\begin{center}
\textsc{\Large Universit\`a degli Studi di Perugia}\medskip\\

{\Large Facolt\`a di Scienze Matematiche, Fisiche e Naturali}\medskip\\

\rule{10mm}{0.01mm}\medskip\\

{\small \textsc{Corso di Laurea in Informatica}}\medskip\\

\vspace*{5mm}

\includegraphics[scale=0.2]{immagini/logouni.png}

\Large Tesi di Laurea \par\bigskip

{\large \bf \titolotesi \par}

\bigskip\bigskip

\end{center}\par

\hspace{0.5cm}Laureando\hspace{7.3cm}Relatori\par

\hspace{0.0cm}\emph{\laureando}\hfill\emph{Prof.~Marco Baioletti}\\
\ \ \hspace*{3.0cm}\hfill\emph{Dott.~Emanuele Palazzetti}\\

\begin{center}

\rule{40mm}{0.01mm}\\

Anno Accademico \annoaccademico

\end{center}

\end{titlepage}

%%%% DEDICA
\newpage
\thispagestyle{empty}
\vspace*{2.5cm}
\begin{flushright}
\begin{Large}\emph{\dedica}\end{Large}
\end{flushright}
\frenchspacing
%%%%%% Ringraziamenti (opzionale)
%
%\chapter*{Ringraziamenti}
%Voglio ringraziare ....
%%%%%%%%%%%%%%%%%%%%%%%%%%%%%%%%%

%%%% INDICE
\thispagestyle{empty}
\tableofcontents



\fancyhf{}
\pagestyle{fancy}
%%%% fine prologo

%%%% Inizio corpo tesi

%%%% Introduzione  
\newpage
\pagenumbering{arabic}
%!TEX root = Tesi__Simone_Mariotti.tex
\chapter*{Introduzione}
\fancyhead[R]{\bfseries Introduzione} 	
\fancyfoot[C]{\thepage } 
\addcontentsline{toc}{chapter}{Introduzione}
I primi automi eseguivano azioni preimpostate semplici e ripetitive; erano da considerarsi poco 
più che macchine utensili. Si pensi ai robot delle catene di produzione di un'autovettura, 
eseguono sempre le stesse azioni, si muovono sempre degli stessi centimetri o millimetri 
portandosi sempre nella stessa posizione, eseguono la loro azione programmata 
e passano al successivo compito. Gli algoritmi che governano tali robot, data la 
loro natura e le specifiche che rispettano, non possono essere considerati intelligenti.\\
Intelligente, in ambito informatico, ha una definizione completamente differente 
dall'intelligenza così come la percepiamo in ambito umano. Una macchina non 
possiede creatività, non impara e non immagina. Se sembra possedere una di queste 
qualità è perché è stata programmata in tal senso. Una macchina non può andare oltre 
la sua programmazione, o almeno non ha ancora potuto. Nessuna macchina ad oggi ha superato 
il Test di Turing\footnote{È una condizione proposta da Alan Turing nel 1950 
che una macchina deve soddisfare per essere considerata intelligente.}. Eugene 
Goostman, un software appositamente sviluppato per emulare il pensiero 
di un ragazzo di 13 anni di origine Ucraina e con una limitata conoscenza dell'inglese,
è riuscito ad ``ingannare'' solo il 29\% dei giudici evitando le domande in modo 
credibile per un ragazzo con le caratteristiche dichiarate. Joshua Tenenbaum,
Professore Ph.D. di Scienze Cognitive Computazionali al MIT, in merito ha dichiarato che 
il risultato non è impressionante.\cite{eugene}\\
In informatica un algoritmo è considerato intelligente
se a seguito dell'analisi di un problema sceglie la soluzione più giusta o più efficiente 
sulla base delle informazioni che possiedono.

In questo lavoro di tesi la nostra intelligenza artificiale realizza una forma semplificata del \emph{bin-picking},
tecnica usata in ambito industriale per identificare e prelevare oggetti 
disposti alla rinfusa in un ambiente.\cite{bin-picking} Nel nostro caso si limiterà a due compiti 
``classici'' per le IA\footnote{Intelligenza Artificiale.}, cercare e identificare 
il suo obiettivo. All'avvio del sistema l'app Android che svolge le funzioni di IA,
ci chiederà di selezionare tramite touch screen un colore visibile sul video 
proveniente dalla camera. Una volta selezionato il colore e premuto il pulsante start, 
il robot si muoverà autonomamente alla ricerca di tale colore nell'ambiente. 
Date le caratteristiche dell'algoritmo implementato, l'ambiente di test potrà 
avere qualsiasi forma ed estensione; per stabilire i suoi confini potremo utilizzare 
un materiale a bassa riflessività, come carta nera opaca, oppure un oggetto per 
formare un piano sopraelevato rispetto all'area circostante.
Il robot si aggirerà per l'ambiente evitando ostacoli e senza 
uscire dall'area delimitata finché non troverà un oggetto del colore cercato. 
A questo punto avvertirà di aver raggiunto l'obiettivo e si metterà in attesa di un nuovo 
input da parte dell'utente.        %Il testo dell'introduzione e' in un file esterno: "introduzione.tex"



%%%% CAPITOLI
%\chapter{Titolo capitolo primo}                 %%% OGNI CAPITOLO INIZIERA' CON QUESTE DUE ISTRUZIONI!
%\fancyhead[RO]{\bfseries Titolo capitolo primo} %%%

%!TEX root = Tesi__Simone_Mariotti.tex
\chapter{Visione Artificiale}
\fancyhead[R]{\bfseries Visione Artificiale} 	 	
\fancyfoot[C]{\thepage } 
La \textit{visione artificiale} è l'unione dei procedimenti che permettono di 
creare  un modello del mondo reale in tre dimensioni a partire da numerose 
immagini bidimensionali per cercare di emulare il processo biologico della vista.
Negli umani, così come in molte altre specie, vedere non è solo scattare una 
fotografia bidimensionale mentale di un ambiente, è interpretare le informazioni
ricevute attraverso la retina per analizzare e creare un modello 3D dell'ambiente
 circostante.\\	
Un sistema di visione artificiale, per provare ad avvicinarsi alla capacità visiva
e di interpretazione umana, ha bisogno di numerosi componenti di diversa natura: 
ottici, elettronici e meccanici per acquisire e memorizzare le immagini da elaborare.
Il primo tentativo in tal senso è datato 1883 quando  Paul Gottlieb Nipkow 
inventò il primo sistema 
in grado di trasformare o meglio trasdurre, informazioni visive in un segnale elettrico.\cite{Nipkow1} 
\\Il sistema
si basa su un disco con dei fori praticati lungo una spirale che parte dal centro del disco
e procede verso l'esterno. Il disco ruota ad una velocità costante mentre l'immagine 
inquadrata viene focalizzata da una lente verso i fori in modo che un sensore, 
posto sul retro del disco, possa percepire i cambi di luminosità della porzione di scena
inquadrata e convertire questa variazione in segnali elettrici. 
\\Il ricevitore emetterà luce in base a dei 
segnali elettrici e attraverso un disco identico al primo e in rotazione sincrona 
proietterà un'immagine con tante righe quante sono i fori del disco.\cite{Nipkow2} 
Questo sistema porterà alla creazione della TV meccanica nel 1925.
\begin{figure}[!htb] \center
\includegraphics[width=\textwidth]{immagini/tv_meccanica.png}
\caption{Immagine visibile su una TV meccanica del 1925} 
\end{figure}
Negli anni successivi, con l'introduzione della miniaturizzazione nell'elettronica, 
sono stati compiuti enormi progressi. Da un punto di vista teorico, gli scienziati 
cominciarono ad ispirarsi al corpo umano, una delle migliori ``macchine'' esistenti, 
nel tentativo di riprodurre il suo comportamento. Analizzando da vicino
l'occhio e in particolare la retina hanno scoperto che era formata da minuscoli 
recettori sensibili alla luce collegati, tramite il nervo ottico, al cervello. 
Si è così deciso di creare dei 
micro sensori di luce e formarne un'enorme matrice. Il primo risultato di questi studi 
si ebbe circa 40 anni dopo e fu il sensore CCD\footnote{Charge-Coupled Device}, 
ideato da Willard S. Boyle e George E. Smith nel 1969 presso i Bell Laboratories,
mentre dobbiamo aspettare il 1993 per vedere i primi prototipi funzionanti del 
sensore CMOS\footnote{Complementary Metal-Oxide-Semiconductor} sviluppati presso 
il Jet Propulsion Laboratory, entrambi hanno come elemento base il fotodiodo che 
equivale ad un pixel. Il CCD è un sensore 
analogico\footnote{Il sensore più grande esistente è di 1,4 Gigapixel ed è 
montato sul telescopio Pan-STARRS sviluppato per l'individuazione di meteoriti in
 rotta di collisione con la Terra} che ha bisogno di più energia 
ma offre in genere una qualità superiore un rumore minore ad un costo più elevato.\\
\begin{figure}[!htb] \center
\includegraphics[scale=0.8]{immagini/ccd-cmos.png}
\caption{Due sensori moderni. A sinistra un sensore CMOS a destra uno CCD} 
\end{figure}
Il CMOS è un sensore digitale che offre una buona qualità di immagine ad costo minore;
per contro l'immagine presenta un forte rumore a causa della conversione 
analogico-digitale.\\
La visione artificiale ha principalmente tre utilizzi:
\begin{itemize}
\item \textbf{Ricognizione:} ricercare uno o più oggetti, scelti a priori, e organizzarli in 
insiemi generici o classi mantenendo informazioni riguardo il loro posizionamento 
nella scena. \\Esempio: ricercare in un'immagine o in un video tutte le persone, 
le macchine o gli animali. 
\item \textbf{Identificazione:} identificare una istanza specifica di una classe 
di oggetti. \\Esempio: ricercare in un'immagine o in un video un volto, 
una macchina o un animale specifico.. 
\item \textbf{Individuazione:} cercare una condizione specifica nell'immagine. 
\\Esempio: cercare imperfezioni nelle immagini a raggi X di superfici o materiali.
\end{itemize}
Un campo che ne fa massiccio utilizzo è la modellazione di ambienti 3D a partire 
da due o più immagini 2D; questo è stato reso possibile 
grazie all'enorme aumento della capacità di elaborazione grafica delle GPU.\\
La visione artificiale è stata applicata a molti altri campi nei quali si è verificata 
una vera e propria rivoluzione.
Uno di questi è la medicina nella quale l'introduzione di questa tecnica ha portato
alla realizzazione di radiografie, angiografie,
tomografie e molte altre; in questo modo è possibile identificare anomalie e 
problemi, quali i tumori, che non sarebbero visibili all'occhio umano 
se non in seguito a procedure molto invasive.
Un altro campo è quello del controllo di veicoli autonomi i quali stanno aumentando
ad un ritmo vertiginoso, progressivamente al migliorarsi delle tecniche di 
visione artificiale. Autovetture, droni, robot e carri per il rifornimento 
sono solo degli esempi a cosa può portare la visione artificiale applicata nei settori 
civili e di ricerca.
\begin{figure}[!htb] \center
\includegraphics[scale=0.2]{immagini/amazon-air.png}
\caption{Drone della società statunitense Amazon. Consegnerà prodotti a domicilio in completa autonomia} 
\end{figure}
La nostra tesi si concentra proprio sull'applicazione ad un robot autonomo dei 
concetti di visione artificiale appena descritti.


%!TEX root = Tesi__Simone_Mariotti.tex
\chapter{Componenti del robot}
\section{Hardware}
\subsection{UDOO Quad}
UDOO è un progetto tutto italiano di una piattaforma hardware destinata alla 
generazione dei ``makers'', cioè quelle persone che vogliono realizzare i 
proprie progetti con le tecnologia a basso costo ad oggi disponibili. La 
scheda ha visto la luce dopo una sorprendente campagna di crowdfunding
\footnote{dall'inglese crowd, folla e funding, finanziamento. In italiano finanziamento collettivo.} terminata l'8 Giugno 2013 con 4172 donazioni per un totale di \$641.612 a fronte di \$27.000 richiesti per iniziare la produzione. Per permettere l'utilizzo di librerie e applicazioni computazionalmente pesanti 
come openCV, PureData e altre UDOO monta un processore ARM Freescale i.MX6 
Cortex-A9 Quad core 1GHz che supporta sia Android che Linux. Il tutto è 
completato da una GPU Vivante, 1GB di RAM DDR3, numerose porte di I/O come 
SATA, microfono, audio out, Ethernet, HDMI, USB, connettore per display LVDS 
con touch screen, connettore CSI per camera esterna e connettività bluetooth e 
Wi-Fi. La periferica di ``boot'' è una microSD il che permette un rapido 
passaggio da Linux a Android e viceversa. Quello che però rende veramente 
unica questa piattaforma, e che ne ha fatto la nostra scelta per questo 
progetto di tesi, è la presenza di un Arduino DUE completamente integrato 
nella stessa board. 
E' presente una CPU Atmel SAM3X8E ARM Cortex-M3 \footnote{la stessa di cui 
dispone l'Arduino DUE} e 76 GPIO\footnote{General Purpose Input/Output}, di 
cui 62 digitali e 14 digitali/analogici, disposti per essere perfettamente 
compatibili con la piedinatura dell'Arduino DUE e dell'Arduino UNO Rev.3.

\begin{figure}[!htb] \center
\includegraphics[width=\textwidth]{immagini/udoo_pinout.png}
\caption{Schema piedinatura UDOO} 
\end{figure}

La presenza di un Arduino DUE all'interno della board rende UDOO una scheda di 
prototipazione a tutti gli effetti e apre nuovi interessanti scenari e 
possibilità unendo la versatilità e semplicità di Arduino, la potenza di 
calcolo del Freescale i.MX6 e le numerose periferiche disponibili per Linux o 
Android.\\
Essendo una piattaforma open-source è possibile accedere alla shell del 
sistema operativo come root tramite la porta seriale integrata e modificare a 
piacimento la configurazione del sistema operativo. Arduino è collegato al 
Freescale i.MX6 tramite un bus interno e quindi viene rilevato come una 
normale periferica USB da Linux; su Android la comunicazione tra i due 
dispositivi avviene sullo stesso bus ma usa lo standard USB OTG\footnote{On-The
-Go è una specifica che permettere di agire come host ad un qualsiasi  
dispositivo (tipicamente smartphone e tablet). A differenza dell'USB classico 
l'OTG è driver-less, cioè non necessita l'installazione di driver specifici 
per ogni dispositivo}. L'interconnessione tra l'accessorio Arduino e 
l'applicazione Android è realizzata tramite l'ADK\footnote{Android Development 
Kit} 2012, di cui parleremo più avanti in questo stesso capitolo, che permette 
l'integrazione delle più disparate periferiche a dispositivi Android tramite 
una connessione USB o Bluetooth.
\subsection {Tank Kit}
Per dare la giusta stabilità e manovrabilità al robot si è deciso di usare una
 locomozione a cingoli che richiede solo due motori e permette di ruotare sul 
 posto o comunque in spazi ristretti: la nostra scelta è stata il ``Multi-
 Chassis - Tank Version''. Questa piattaforma, appositamente pensata per la 
 realizzazione di robot multifunzione, si è rivelata la scelta perfetta in 
 quanto possiede due potenti motori DC già forniti di riduttori 48:1 per 
 affrontare terreni impervi e scoscesi, quattro ruote da 52mm di diametro a 
 cui sono applicati i due cingoli.E' presente anche un alloggiamento per un 
 servomotore standard che nella nostra applicazione non è stato usato. 
 L'intelaiatura, di alluminio spesso 2,5mm, presenta numerosi fori e asole per 
 il montaggio di accessorie quali sensori, staffe e motori. Presenta inoltre 
 un ``doppio fondo'' in cui sono alloggiati i motori DC e i riduttori e in cui 
 è possibile sistemare altri componenti che non debbano essere facilmente 
 accessibili.
\subsection {Sensori}
\subsubsection{Sensore di riflessività - QRD1114}
Avevamo la necessità di fornire al robot un modo per rilevare eventuali 
sconfinamenti dall'ambiente di test che fosse il più flessibile possibile. 
Abbiamo optato per il sensore di riflessività QRD1114 prodotto dalla Fairchild 
Semiconductor: questo sensore è costituito da un LED infrarosso e un 
fototransistor tarato sulla luce infrarossa e con filtro per la luce solare 
onde evitare disturbi. Il robot era stato pensato per lavorare su un tavolo o 
altra superficie con spigoli netti: per rilevare l'imminente caduta in questo 
tipo di ambiente sarebbe stato sufficiente un sensore di distanza puntato 
verso terra. Con il sensore di riflessività abbiamo reso possibile l'utilizzo 
in terra o comunque in ambienti estesi delimitati da un recinto spesso circa 
10cm realizzato con materiale a basse riflettività come del semplice 
cartoncino nero opaco. Il sensore non fa differenza tra il cartoncino nero o 
lo spazio a fianco di un tavolo, rileva semplicemente una riflessività vicina 
allo zero. Il sensore così come fornito dal produttore non è direttamente 
utilizzabile, per far si che Arduino potesse acquisire dal sensore valori 
proporzionali alla riflessività del materiale in esame abbiamo dovuto 
realizzare un circuito elettronico di interfaccia. 
\begin{figure}[!htb] \center
\includegraphics[ scale=0.2]{immagini/QRD1114.png}
\caption{Circuito di interfaccia tra Arduino e il sensore QRD1114} 
\end{figure}
Il circuito alimenta il LED tramite una resistenza da 220$\Omega$ (\textit{R1}) 
e collega $V_{CC}$\footnote{pari a 3,3 $V$ nell'Arduino DUE} al collettore del 
fototransistor tramite una resistenza da 10 $k\Omega$ (\textit{R2}) mentre 
l'emettitore è collegato a terra; il punto da cui prelevare il segnale ($V_{OUT
}$) è tra \textit{R2} e il collettore. Il principio alla base del circuito è 
semplice: il LED è sempre acceso e illumina in modo diffuso parallelamente al 
fototransistor. Il fototransistor in assenza di luce o, nel nostro caso, in 
presenza di una bassa riflessività si trova in stato di interdizione; i pin di 
Arduino impostati come input sono in configurazione ``alta impedenza'', 
equivalenti ad un interruttore aperto dal punto di vista circuitale, quindi 
non c'è passaggio di corrente né tramite il transistor né tramite la 
resistenza \textit{R2} il che porta esattamente il valore di $V_{cc}$ in 
ingresso ad Arduino. Quando il fototransistor è totalmente illuminato, cioè in 
presenza di alta riflessività, entra in stato di conduzione così che nel punto 
$V_{OUT}$ si venga a trovare la massa. Ogni stato intermedio di illuminazione 
equivale ad una conduzione parziale del fotoresistore a cui corrisponde una 
tensione proporzionale alla riflessività sul pin $V_{OUT}$. Il sensore può 
essere utilizzato in modalità digitale o analogica semplicemente collegando il 
pin $V_{OUT}$ ad un pin digitale o analogico e cambiando la configurazione 
relativa all'interno della programmazione di Arduino. La configurazione 
digitale si è rivelata inadatta all'applicazione in quanto l'intervallo di 
valori che rappresenta un oggetto riflette nelle immediate vicinanze del 
sensore, e quindi lo stato di normale funzionamento, è troppo stretto e anche 
un minimo sussulto manda il robot in allarme sconfinamento. La configurazione 
analogica invece ci permette di avere circa 1024 valori discreti dal 
trasduttore, abbiamo quindi impostato una soglia oltre la quale il robot va in 
allarme sconfinamento; è importante notare che questa libertà nello scegliere 
una soglia di allarme ci permette di effettuare una calibrazione affinata in 
base al materiale su cui si svolge il test per minimizzare la possibilità di 
falsi positivi. 
\section{Software}
\subsection {OpenCV}
\subsection {ADK}
\subsection {ADK Toolkit}

%!TEX root = Tesi__Simone_Mariotti.tex
\chapter{Componenti software}
\fancyhead[R]{\bfseries Componenti software}     
\fancyfoot[C]{\thepage } 
\section {OpenCV}
Per dare al robot una visione dell'ambiente circostante ci serviva una libreria di 
visione artificiale. Le più famose disponibili per Java sono OpenCV e JavaCV; 
la nostra scelta è stata OpenCV per via della maggiore disponibilità di documentazione. 
OpenCV\footnote{Open Source Computer Vision} è stata originariamente sviluppata 
da Intel per migliorare le prestazioni delle loro CPU con applicazioni
computazionalmente pesanti come per esempio gli algoritmi di \textit{ray tracing}.
\footnote{Tecnica di geometrica ottica che analizza il percorso dei raggi di luce.
In grafica 3D è un algoritmo di rendering che costruisce la scena facendo
partire i raggi luminosi dalla camera (visuale del giocatore) invece che dalla 
sorgente luminosa.\cite{ray_tracing}} Questa libreria è rilasciata sotto licenza BSD. 
\\La libreria per poter funzionare su un 
dispositivo Android ha bisogno di un'applicazione di supporto chiamata OpenCV 
Manager che può essere scaricata dal Play Store o installata direttamente con il
pacchetto APK.\footnote{Android Package, il tipo di file delle app Android simile 
al formato JAR } La libreria viene caricata in modo asincrono all'avvio dell'app
e fornisce ogni frame come una matrice dove ogni elemento rappresenta un pixel; 
OpenCV usa il tipo di dato interno Mat per rappresentare le matrici.
Tutte le operazioni effettuate su oggetti di tipo Mat usano algoritmi appositamente 
ottimizzati per le operazioni tra matrici. 

\section {ADK}
In Android, a partire dalla versione 2.3.4, è supportato il protocollo Android Open 
Accessory (AOA). Ogni accessorio sviluppato per Android deve comunicare con il sistema operativo 
tramite lo stesso protocollo. 
\\A causa della limitata autonomia dei dispositivi su cui Android è maggiormente 
utilizzato (smarthphone e tablet), l'AOA impone che l'accessorio funga da host 
e alimenti quindi il bus utilizzato.\cite{aoa}
\\A partire dal 2011, Google ha rilasciato open source gli schemi e l'implementazione 
della libreria ADK per tutti i dispositivi Arduino compatibili, così da fornire 
un'interfaccia comune per garantire la comunicazione con i dispositivi Android, 
previa un'opportuna configurazione. Android distingue diversi accessori 
sulla base delle informazioni da loro forniti durante la connessione preliminare. 
\\Queste informazioni possono essere distinte nei seguenti valori:
\begin{itemize}
\item Vendor
\item Name
\item Longname
\item Version
\item Url
\item Serial
\end{itemize} 
Il dispositivo Android accetterà la richiesta di connessione solo se le stringhe 
identificative in suo possesso combaciano con quelle fornite 
dall'accessorio. Per ottenere effettivamente la connessione bisogna indicare ad 
Android quali sono gli accessori supportati e quale applicazioni è in grado di 
gestire un particolare accessorio. Questo si ottiene inserendo la successiva 
direttiva nel Manifest file dell'app

\lstset{language=XML}

\begin{lstlisting}[caption=Porzione del Manifest file dell'app]
...
<meta-data
    android:name="android.hardware.usb.action.
    				USB_ACCESSORY_ATTACHED"
    android:resource="@xml/usb_accessory_filter" />
...
\end{lstlisting}
Alla connessione di qualsiasi accessorio, il sistema operativo sarà incaricato 
di suggerire ed aprire l'apposita app in grado di gestire l'accessorio connesso. 
L'associazione con un particolare accessorio viene verificato tramite il file 
usb\_accessory\_filter.xml associato.
\\Nel nostro caso il file usb\_accessory\_filter.xml si presenta così:
\begin{lstlisting}[caption=usb\_accessory\_filter.xml]
<resources>
    <usb-accessory
            version="0.1.0"
            model="Mobile-Tanker"
            manufacturer="Simone Mariotti"/>
</resources>
\end{lstlisting}
Le stesse identiche stringhe saranno impostate durante la fase di configurazione 
di Arduino in modo da permette l'accoppiamento.
\section {ADK Toolkit}
L'ADK toolkit è una libreria che aggiunge un grado di astrazione all'ADK, semplificando 
l'inizializzazione della connessione, l'invio e la ricezione dei messaggi.\\
Il toolkit si basa su due classi principali: AdkManager e AdkMessage.	\\	
AdkManager espone metodi per la gestione della connessione e per l'invio e la 
ricezioni di dati. AdkMessage rappresenta il messaggio ricevuto dall'accessorio 
nel suo formato nativo, cioè un array di byte; tramite dei metodi ausiliari, 
\textit{getString()}, \textit{getFloat()}, \textit{getInt()}, è possibile 
ottenere il typecast del messaggio ricevuto, normalmente rappresentato da un array di byte. 
Il recupero del messaggio originale può essere ottenuto grazie ai metodi 
\textit{getBytes()} e \textit{getByte()}.\\
Grazie a questa libreria l'uso dell'ADK, originariamente poco intuitivo, diventa 
efficiente ed elegante
\lstset{
language=Java,
frame=tb,  
  aboveskip=3mm,
  belowskip=3mm,
  showstringspaces=false,
  columns=flexible,
  basicstyle={\small\ttfamily},
  numbers=none,
  identifierstyle=\color{black},
  numberstyle=\tiny\color{gray},
  keywordstyle=\color{blue},
  commentstyle=\color{dkgreen},
  stringstyle=\color{mauve},
  breaklines=true,
  breakatwhitespace=true,
  tabsize=3
}
\begin{lstlisting}[caption=Inizializzazione della connessione con l'accessorio]
private AdkManager mAdkManager;

@Override
protected void onCreate(Bundle savedInstanceState) {
    ...
    mAdkManager = new AdkManager(this);
}

@Override
protected void onResume() {
    super.onResume();
    mAdkManager.open();
}
\end{lstlisting}

\begin{lstlisting}[caption=Lettura e scrittura dati]
// Write
adkManager.write("Ciao da Android!");

// Read
AdkMessage response = adkManager.read();
System.out.println(response.getString());
// Esempio di output: "Ciao da Arduino!"
\end{lstlisting}
\begin{lstlisting}[caption=Chiusura della connessione con l'accessorio]
@Override
protected void onDestroy() {
    ...
    mArduino.close();
}
\end{lstlisting}
%!TEX root = Tesi__Simone_Mariotti.tex
\chapter{Implementazione}
\fancyhead[R]{\bfseries Implementazione} 	
\fancyfoot[C]{\thepage }
\section {Android}
L'app per Android è stata sviluppata avendo come priorità la modularità. E' 
infatti molto semplice cambiare totalmente il comportamento del robot o la 
codifica dei messaggi sostituendo o modificando una singola classe senza 
coinvolgere il resto del codice.\\
La modularizzazione a grana più grossa è a livello di package\footnote{Un contenitore 
che racchiude classi che svolgono compitini affini}: ci sono tre package,
ognuno con un compito preciso, e sono \textbf{logic}, \textbf{messaging} e \textbf{opencv}.
\begin{figure}[H] \center
\includegraphics[scale=0.2]{immagini/package_diagram.png}
\caption{Schema di interconnessione dei package} 
\end{figure}
\subsection {Il package \textit{logic}}
Si occupa di reperire le informazioni dal mondo reale e analizzarle al fine di
eseguire l'azione più adatta.
Contiene tre classi: RobotActivity, RobotLogic e UpdateDirections.
\subsubsection{La classe \emph{RobotActivity}}
Come suggerisce il nome è l'Activity vera e propria, cioè quella classe che il 
sistema operativo istanzia all'avvio dell'app. Essa stessa istanzia e prepara tutti gli 
altri oggetti per l'esecuzione. Implementa due interfacce: \textit{View.OnTouchListener} 
e \textit{CvCameraViewListener}. 
La prima permette di gestire gli input da touch screen senza ricorrere ad una classe esterna,
la seconda è un'interfaccia presente nella libreria OpenCV e permette di ``intercettare''
i frame proveniente dalla camera prima che vengano renderizzati a schermo tramite 
l'override\footnote{Tecnica che permette di ridefinire il comportamento di un metodo 
ereditato} del metodo \textit{OnCameraFrame()}. Ogni frame verrà elaborato e solo alla
fine visualizzato a schermo.
All'avvio si occupa di inizializzare OpenCV tramite la 
callback \textit{BaseLoaderCallback} e ottiene il riferimento all'istanza dell'ADK
tramite l'ADKToolkit. E' buona norma interrompere le connessioni e liberare il canale usato 
nel momento in cui un'app viene chiusa. Per questo nel metodo \emph{onDestroy()} 
viene chiuso il canale usato per comunicare con Arduino tramite il metodo \emph{close()}
fornito dall'ADKToolkit.

 \subsubsection{La classe \emph{UpdateDirections}}
 Questa classe è un ``singleton''\footnote{E' un design pattern descritto dalla 
 cosiddetta ``Gang of four'' nel libro ``Design Patterns''. 
 Permette la creazione di una sola istanza della classe e ne regola l'accesso. } 
 e si occupa di mostrare all'utente tramite  immagini e testi quello che  il 
 robot sta facendo o quale sarà la sua prossima  mossa. Dovendo agire 
 sull'UI\footnote{User Interface, interfaccia utente} deve essere eseguita sul thread
 che si occupa dell'UI, per questo implementa l'interfaccia \emph{Runnable} e 
 ogni sua esecuzione è lanciata  tramite \emph{runOnUiThread()}. 
 Le informazioni visualizzabili sono limitate e ben distinte,
 ad ognuna corrisponde un metodo da invocare per visualizzare quella data informazione
 a schermo. I metodi disponibili sono: 
 \begin{itemize}
 \item \emph{left()}: indica che l'obiettivo è visibile è stato individuato e si trova alla sinistra del robot.
 \item \emph{right()}: indica che l'obiettivo è visibile è stato individuato e si trova alla destra del robot.
 \item \emph{aimed()}: indica che l'obiettivo è visibile è stato individuato e si trova esattamente di fronte al robot che si muoverà in quella direzione.
 \item \emph{search()}: indica che l'obiettivo non è visibile, il robot si muoverà secondo l'algoritmo di ricerca.
 \item \emph{found()}: indica che l'obiettivo è visibile e il robot si trova a meno di 30 centrimetri.
 \item \emph{avoidingLeft()}: indica che è presente un ostacolo sul percorso del robot a meno di 30 centimetri. Il robot lo aggirerà verso sinistra. 
 \item \emph{avoidingRight()}: indica che è presente un ostacolo sul percorso del robot a meno di 30 centimetri. Il robot lo aggirerà verso destra.
 \item \emph{chooseColor()}: indica che il robot è in attesa che venga impostato il colore da cercare. 
 \end{itemize}
 La classe espone anche altri metodi che mostrano (\emph{show()}) o nascondono 
 (\emph{hide()}) la porzione di interfaccia che visualizza le indicazioni, oppure bloccano (\emph{lock()})
 e sbloccano (\emph{unlock()}) la possibilità di cambiare le indicazioni visualizzate.

\subsubsection{La classe \emph{RobotLogic}}
La classe \emph{RobotLogic} è il vero ``cervello'' di tutta l'app.\\
Viene chiamata in causa ogni volta che arriva un nuovo messaggio da Arduino. 
Per far questo si è usato un altro famoso design pattern, quello di ``Observable'' 
e ``Observer'' in cui l'``Observer'' è questa stessa classe e l'``Observable''
è la classe IncomingMessage del pacchetto \textbf{messaging}.\\
Alla ricezione di ogni messaggio viene invocato il metodo \emph{decodeMessage()}
che decodifica il messaggio ricevuto e crea un'istanza di DecodedMessage per 
accogliere le informazioni appena ricevute e facilitarne l'accesso. Se il messaggio
contiene informazioni sulla distanza del primo oggetto nella direzione del robot
allora tale distanza viene salvata nell'istanza in modo che successivamente si
possa usare per valutare in modo più preciso la situazione.\\
Il metodo principale della classe è \emph{think()} ed è invocato ogni volta che
dal package opencv, e in particolare dalla classe TargetSearch, giunge un aggiornamento
sull'obbiettivo, la sua eventuale presenza in scena e la sua posizione. \emph{think()}
viene effettivamente eseguito solo se sull'interfaccia utente è stato premuto "Start"
 e per prima cosa stabilisce in che fase l'iterazione precedente ha portato il sistema. \\
 Le possibilità sono:
	\begin{itemize}
	\item ``Cheer Phase'': indica che l'obbiettivo è stato trovato e il robot sta
	girando su se stesso per segnalare la fine della ricerca.
	\item ``Avoiding Phase'': indica che sulla traiettoria del robot si è presentato un ostacolo e lo sta aggirando. Si compone di tre sottofasi:
	\begin{itemize}
		\item Fase 1: Scegliere in modo casuale una direzione tra destra e sinistra e ruota di circa 90° in quella direzione. 
		\item Fase 2: Si muove in avanti per un tempo prestabilito.
		\item Fase 3: Ruota di 90° nella direzione opposta a quella scelta nella fase 1.
	\end{itemize}
	\item ``Search Phase'': indica che l'obbiettivo non è stato ancora trovato e il robot si sta muovendo per trovarlo.
	\end{itemize}

Se si trova nella ``Cheer Phase'' ignora ogni comunicazione proveniente da 
Arduino, finisce la rotazione di segnalazione e invoca il metodo \emph{reset()} di 
RobotActivity per preparare il sistema all'inserimento di un nuovo obbiettivo da 
parte dell'utente.\\
Se si trova nella ``Avoiding Phase'' esegue in successione le tre fasi. 
Si può interrompere solo se durante le manovre di aggiramento dell'ostacolo
l'obbiettivo compare nella scena.\\
Se si trova nella ``Search Phase'' controlla se l'obbiettivo è presente in scena e 
si trova a meno di 30 centimetri, in caso di risposta affermativa ferma il robot,
interrompe l'esecuzione di \emph{think()} e attiva la ``Cheer Phase''. 
Se l'obbiettivo non è in vista ma c'è un oggetto a meno di 30 centimetri ferma il robot
e attiva la ``Avoiding Phase''.\\
Se l'obbiettivo non è visibile rimane in ``Search Phase'' e si muove in avanti.\\
Se l'obbiettivo è visibile ma si trova a più di 30 centimetri effettua le correzioni di rotta
necessarie per portarlo in direzione di marcia.\\
Se l'obbiettivo è precisamente di fronte al robot si muoverà in quella direzione.\\ 
La rotazione necessaria per portare l'obbiettivo esattamente di fronte al robot 
implica movimenti lenti e precisi che è difficile ottenere con i motori DC di cui è
fornito il robot. Lo scenario tipico è vedere i motori sforzarsi di muovere il robot
con un certo quantitativo di energia, aumentare quell'energia di un'unità e vedere 
il robot iniziare a ruotare velocemente. Purtroppo l'energia richiesta per 
mettere in rotazione il robot cambiava in funzione di troppe variabili: 
la direzione di rotazione, l'uso di entrambi i cingoli o solamente uno, 
la presenza di piccoli ostacoli sotto il robot. Per ovviare a questo problema che impediva 
una corretta movimentazione del robot quando erano necessari movimenti precisi 
si è usato una soluzione adattiva che permette al robot di trovare sempre la 
minima energia necessaria per muoversi. Per far questo il robot prende come riferimento 
la posizione dell'obbiettivo e tenta di ruotare, se all'iterazione successiva 
non nota uno spostamento relativamente all'obbiettivo allora aumenta 
l'energia inviata ad i motori. Appena si muove l'energia cessa di essere 
aumentata e il robot ruota in modo fluido e controllato.
\subsection {Il package \textit{opencv}}


 



\section {Arduino} 



%%%% CONCLUSIONI   (non numerate come per l'introduzione) 
\newpage
%!TEX root = Tesi__Simone_Mariotti.tex
\chapter*{Conclusioni e sviluppi futuri}
\addcontentsline{toc}{chapter}{Conclusioni}
\fancyfoot[C]{\thepage } 



%%%% BIBLIOGRAFIA
\newpage 
\begin{thebibliography}{10}
\fancyfoot[C]{\thepage } 
\addcontentsline{toc}{chapter}{Bibliografia}
\end{thebibliography}
     %conterra' una serie di istruzioni del tipo:    
                         % \bibitem{eol} O. Gervasi and A. Lagan‡ "EOL: A Web-Based Distance Assessment System", 
			 % Lecture Notes in Computer Science, 3044, Springer & Verlag, pp. 854-862 (2004)



%%%% ELENCO DELLE IMMAGINI
\newpage
\listoffigures
\addcontentsline{toc}{chapter}{Elenco delle immagini}



%%%% APPENDICE CON IL CODICE SVILUPPATO
\appendix
\linespread{1}
%!TEX root = Tesi__Simone_Mariotti.tex
\chapter*{Appendice}
%%%%%%%%%%%%%%%%%%%
%%%%% ESEMPIO:
%\section{index.php}
%\begin{footnotesize}
%\begin{verbatim}
% .... segue file con il codice  ...

\end{document}


