%!TEX root = Tesi__Simone_Mariotti.tex
\chapter*{Introduzione}
\fancyhead[R]{\bfseries Introduzione} 	
\fancyfoot[C]{\thepage } 
\addcontentsline{toc}{chapter}{Introduzione}
I primi automi eseguivano azioni preimpostate semplici e ripetitive, erano poco 
più che macchine utensili. Si pensi ai robot delle catene di produzione di un'autovettura, 
i robot eseguono sempre le stesse mosse, si muovono sempre della stesa quantità 
e si portano sempre nella stessa posizione, eseguono la loro azione programmata 
e passano al successivo compito. Di sicuro non possono essere considerati intelligenti.\\
Intelligente in ambito robotico ha una definizione completamente differente 
dall'intelligenza così come la percepiamo in ambito umano. Una macchina non 
possiede creatività, non impara e non immagina. Se sembra possedere una di queste 
qualità è perché è stata programmata in tal senso. Una macchina non può andare oltre 
la sua programmazione, o almeno non ha ancora potuto. Nessuna macchina ha superato 
il Test di Turing\footnote{è una condizione proposta da Alan Turing nel 1950 
che una macchina deve soddisfare per essere considerata intelligente.}, tranne 
una: Eugene Goostman, un software appositamente sviluppato per emulare il pensiero 
di un ragazzo di 13 anni e superare tale test. 
L'intelligenza di una macchina ha quindi come limite invalicabile solo quella 
del suo programmatore.\\
In robotica e in informatica una macchina o un algoritmo sono considerati intelligenti
se a seguito di un problema scelgono la soluzione più giusta o più efficiente.

In questo lavoro di tesi la nostra intelligenza artificiale realizza una forma semplificata del \emph{bin-picking},
tecnica usata in ambito industriale per identificare e prelevare oggetti 
disposti alla rinfusa in un ambiente. Nel nostro caso si limita a due compiti 
``classici'' per le IA\footnote{Intelligenza Artificiale}, cerca e identifica 
il suo obbiettivo. All'avvio del sistema l'app Android che svolge le funzioni di IA
ci chiede di selezionare tramite touch screen un colore visibile sul video 
proveniente dalla camera. Una volta selezionato il colore e premuto il pulsante start 
il robot si muove autonomamente alla ricerca di tale colore nell'ambiente. L'ambiente di test
può avere qualsiasi forma ed estensione; per stabilire i suoi confini deve essere usato un 
materiale a bassa riflessività come carta nera opaca oppure deve essere sopraelevato rispetto 
all'area circostante. Il robot si aggira per l'ambiente evitando ostacoli e senza 
uscire dall'area delimitata finché non trova un oggetto del colore cercato. 
A questo punto avverte di aver raggiunto l'obbiettivo e si mette in attesa di un nuovo 
input da parte dell'utente.

Particolarmente interessante è che l'architettura software utilizzata permette 
di cambiare l'obbiettivo da cercare in qualunque altra cosa, che sia un viso, un 
forma particolare o un oggetto specifico semplicemente modificando o sostituendo il modulo che si occupa 
del riconoscimento delle immagini senza dover minimamente alterare il resto del 
codice che svolge funzioni di IA o comunicazione.


