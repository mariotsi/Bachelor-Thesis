%!TEX root = Tesi__Simone_Mariotti.tex
\chapter{Componenti software}
\fancyhead[R]{\bfseries Componenti software}     
\fancyfoot[C]{\thepage } 
\section {OpenCV}
Per dare al robot una visione dell'ambiente circostante ci serviva una libreria di 
visione artificiale. Le più famose disponibili per Java sono OpenCV e JavaCV; 
la nostra scelta è stata OpenCV per via della maggiore disponibilità di documentazione. 
OpenCV\footnote{Open Source Computer Vision} è stata originariamente sviluppata 
da Intel per migliorare le prestazioni delle loro CPU con applicazioni
computazionalmente pesanti come per esempio gli algoritmi di \textit{ray tracing}.
\footnote{Tecnica di geometrica ottica che analizza il percorso dei raggi di luce.
In grafica 3D è un algoritmo di rendering che costruisce la scena facendo
partire i raggi luminosi dalla camera (visuale del giocatore) invece che dalla 
sorgente luminosa.\cite{ray_tracing}} Questa libreria è rilasciata sotto licenza BSD. 
\\La libreria per poter funzionare su un 
dispositivo Android ha bisogno di un'applicazione di supporto chiamata OpenCV 
Manager che può essere scaricata dal Play Store o installata direttamente con il
pacchetto APK.\footnote{Android Package, il tipo di file delle app Android simile 
al formato JAR } La libreria viene caricata in modo asincrono all'avvio dell'app
e fornisce ogni frame come una matrice dove ogni elemento rappresenta un pixel; 
OpenCV usa il tipo di dato interno Mat per rappresentare le matrici.
Tutte le operazioni effettuate su oggetti di tipo Mat usano algoritmi appositamente 
ottimizzati per le operazioni tra matrici. 

\section {ADK}
In Android, a partire dalla versione 2.3.4, è supportato il protocollo Android Open 
Accessory (AOA). Ogni accessorio sviluppato per Android deve comunicare con il sistema operativo 
tramite lo stesso protocollo. 
\\A causa della limitata autonomia dei dispositivi su cui Android è maggiormente 
utilizzato (smarthphone e tablet), l'AOA impone che l'accessorio funga da host 
e alimenti quindi il bus utilizzato
\\A partire dal 2011, Google ha rilasciato open source gli schemi e l'implementazione 
della libreria ADK per tutti i dispositivi Arduino compatibili, così da fornire 
un'interfaccia comune per garantire la comunicazione con i dispositivi Android, 
previa un'opportuna configurazione. Android distingue diversi accessori 
sulla base delle informazioni da loro forniti durante la connessione preliminare. 
\\Queste informazioni possono essere distinte nei seguenti valori:
\begin{itemize}
\item Vendor
\item Name
\item Longname
\item Version
\item Url
\item Serial
\end{itemize} 
Il dispositivo Android accetterà la richiesta di connessione solo se le stringhe 
identificative in suo possesso combaciano con quelle fornite 
dall'accessorio. Per ottenere effettivamente la connessione bisogna indicare ad 
Android quali sono gli accessori supportati e quale applicazioni è in grado di 
gestire un particolare accessorio. Questo si ottiene inserendo la successiva 
direttiva nel Manifest file dell'app

\lstset{language=XML}

\begin{lstlisting}[caption=Porzione del Manifest file dell'app]
...
<meta-data
    android:name="android.hardware.usb.action.
    				USB_ACCESSORY_ATTACHED"
    android:resource="@xml/usb_accessory_filter" />
...
\end{lstlisting}
Alla connessione di qualsiasi accessorio, il sistema operativo sarà incaricato 
di suggerire ed aprire l'apposita app in grado di gestire l'accessorio connesso. 
L'associazione con un particolare accessorio viene verificato tramite il file 
usb\_accessory\_filter.xml associato.
\\Nel nostro caso il file usb\_accessory\_filter.xml si presenta così:
\begin{lstlisting}[caption=usb\_accessory\_filter.xml]
<resources>
    <usb-accessory
            version="0.1.0"
            model="Mobile-Tanker"
            manufacturer="Simone Mariotti"/>
</resources>
\end{lstlisting}
Le stesse identiche stringhe saranno impostate durante la fase di configurazione 
di Arduino in modo da permette l'accoppiamento.
\section {ADK Toolkit}
L'ADK toolkit è una libreria che aggiunge un grado di astrazione all'ADK, semplificando 
l'inizializzazione della connessione, l'invio e la ricezione dei messaggi.\\
Il toolkit si basa su due classi principali: AdkManager e AdkMessage.	\\	
AdkManager espone metodi per la gestione della connessione e per l'invio e la 
ricezioni di dati. AdkMessage rappresenta il messaggio ricevuto dall'accessorio 
nel suo formato nativo, cioè un array di byte; tramite dei metodi ausiliari, 
\textit{getString()}, \textit{getFloat()}, \textit{getInt()}, è possibile 
ottenere il typecast del messaggio ricevuto, normalmente rappresentato da un array di byte. 
Il recupero del messaggio originale può essere ottenuto grazie ai metodi 
\textit{getBytes()} e \textit{getByte()}.\\
Grazie a questa libreria l'uso dell'ADK, originariamente poco intuitivo, diventa 
efficiente ed elegante
\lstset{
language=Java,
frame=tb,  
  aboveskip=3mm,
  belowskip=3mm,
  showstringspaces=false,
  columns=flexible,
  basicstyle={\small\ttfamily},
  numbers=none,
  identifierstyle=\color{black},
  numberstyle=\tiny\color{gray},
  keywordstyle=\color{blue},
  commentstyle=\color{dkgreen},
  stringstyle=\color{mauve},
  breaklines=true,
  breakatwhitespace=true,
  tabsize=3
}
\begin{lstlisting}[caption=Inizializzazione della connessione con l'accessorio]
private AdkManager mAdkManager;

@Override
protected void onCreate(Bundle savedInstanceState) {
    ...
    mAdkManager = new AdkManager(this);
}

@Override
protected void onResume() {
    super.onResume();
    mAdkManager.open();
}
\end{lstlisting}

\begin{lstlisting}[caption=Lettura e scrittura dati]
// Write
adkManager.write("Ciao da Android!");

// Read
AdkMessage response = adkManager.read();
System.out.println(response.getString());
// Esempio di output: "Ciao da Arduino!"
\end{lstlisting}
\begin{lstlisting}[caption=Chiusura della connessione con l'accessorio]
@Override
protected void onDestroy() {
    ...
    mArduino.close();
}
\end{lstlisting}