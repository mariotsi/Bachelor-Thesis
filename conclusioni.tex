%!TEX root = Tesi__Simone_Mariotti.tex
\chapter*{Conclusioni}
\addcontentsline{toc}{chapter}{Conclusioni}
\fancyfoot[C]{\thepage}
Il robot riesce a distinguere colori che siano distanti tra loro 10$^\circ$ nel modello di colori HSV,
il che permette di cercare 36 colori differenti. 

Riesce a trovare adeguatamente 
l'obiettivo seppur senza seguire un percorso efficiente data l'assenza di conoscenza 
dell'ambiente di sperimentazione. Molte situazioni di stallo si risolvono grazie alla randomizzazione 
di alcune parti del processo decisionale mentre altre, perlopiù avvenute in presenza di due 
oggetti nelle vicinanze dei bordi dell'ambiente, hanno richiesto un intervento esterno per la soluzione. 
L'affidabilità dimostrata nella ricerca è comunque tale che riteniamo tutti gli obiettivi 
prefissati per questo lavoro di tesi raggiunti in modo ottimale.

La grande modularità del software ha di fatto creato un semplice framework 
per l'implementazione di nuovi algoritmi di ricerca \emph{context-aware}\footnote{Che sono a conoscenza del contesto in cui operano.}, 
da usare con una qualsiasi piattaforma composta da un'app Android ed un Arduino Due. 
Come visto nel Capitolo 4, l'estensione della
classe BaseAi e l'implementazione del metodo \emph{think()} è tutto ciò che serve per
creare un algoritmo di IA funzionante.
\subsection* {Sviluppi futuri}
Il software è stato rilasciato con licenza BSD è quindi migliorabile, 
modificabile e personalizzabile da futuri studenti.
\\Miglioramenti possibili sono:
\begin{itemize}
\item aggiungere sensori per la navigazione come GPS, 
bussole e giroscopi. Permetterebbero di eliminare la casualità nella navigazione e seguire un efficiente percorso di ricerca dell'obiettivo.
\item implementare una base di conoscenza in grado di tenere traccia degli ostacoli e degli oggetti riconosciuti in modo da velocizzare ogni successiva ricerca.
\item aggiungere un attuatore in grado di manipolare l'obiettivo una volta individuato.
\end{itemize}

