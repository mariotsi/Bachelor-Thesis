%!TEX root = Tesi__Simone_Mariotti.tex
\chapter{Visione Artificiale}
\fancyhead[R]{\bfseries Visione Artificiale} 	 	
\fancyfoot[C]{\thepage } 
La \textit{visione artificiale} è l'unione dei procedimenti che permettono di 
creare  un modello del mondo reale in tre dimensioni a partire da numerose 
immagini bidimensionali per cercare di emulare il processo biologico della vista.
Negli umani, così come in molte altre specie, vedere non è solo scattare una 
fotografia bidimensionale mentale di un ambiente, è interpretare le informazioni
ricevute attraverso la retina per analizzare e creare un modello 3D dell'ambiente
 circostante.\\	
Un sistema di visione artificiale, per provare ad avvicinarsi alla capacità visiva
e di interpretazione umana, ha bisogno di numerosi componenti di diversa natura: 
ottici, elettronici e meccanici per acquisire e memorizzare le immagini da elaborare.
Il primo tentativo in tal senso è datato 1883 quando  Paul Gottlieb Nipkow 
inventò il primo sistema 
in grado di trasformare o meglio trasdurre, informazioni visive in un segnale elettrico.\cite{Nipkow1} 
\\Il sistema
si basa su un disco con dei fori praticati lungo una spirale che parte dal centro del disco
e procede verso l'esterno. Il disco ruota ad una velocità costante mentre l'immagine 
inquadrata viene focalizzata da una lente verso i fori in modo che un sensore, 
posto sul retro del disco, possa percepire i cambi di luminosità della porzione di scena
inquadrata e convertire questa variazione in segnali elettrici. 
\\Il ricevitore emetterà luce in base a dei 
segnali elettrici e attraverso un disco identico al primo e in rotazione sincrona 
proietterà un'immagine con tante righe quante sono i fori del disco.\cite{Nipkow2} 
Questo sistema porterà alla creazione della TV meccanica nel 1925.
\begin{figure}[!htb] \center
\includegraphics[width=\textwidth]{immagini/tv_meccanica.png}
\caption{Immagine visibile su una TV meccanica del 1925} 
\end{figure}
Negli anni successivi, con l'introduzione della miniaturizzazione nell'elettronica, 
sono stati compiuti enormi progressi. Da un punto di vista teorico, gli scienziati 
cominciarono ad ispirarsi al corpo umano, una delle migliori ``macchine'' esistenti, 
nel tentativo di riprodurre il suo comportamento. Analizzando da vicino
l'occhio e in particolare la retina hanno scoperto che era formata da minuscoli 
recettori sensibili alla luce collegati, tramite il nervo ottico, al cervello. 
Si è così deciso di creare dei 
micro sensori di luce e formarne un'enorme matrice. Il primo risultato di questi studi 
si ebbe circa 40 anni dopo e fu il sensore CCD\footnote{Charge-Coupled Device}, 
ideato da Willard S. Boyle e George E. Smith nel 1969 presso i Bell Laboratories,
mentre dobbiamo aspettare il 1993 per vedere i primi prototipi funzionanti del 
sensore CMOS\footnote{Complementary Metal-Oxide-Semiconductor} sviluppati presso 
il Jet Propulsion Laboratory, entrambi hanno come elemento base il fotodiodo che 
equivale ad un pixel. Il CCD è un sensore 
analogico\footnote{Il sensore più grande esistente è di 1,4 Gigapixel ed è 
montato sul telescopio Pan-STARRS sviluppato per l'individuazione di meteoriti in
 rotta di collisione con la Terra} che ha bisogno di più energia 
ma offre in genere una qualità superiore un rumore minore ad un costo più elevato.\\
\begin{figure}[!htb] \center
\includegraphics[scale=0.8]{immagini/ccd-cmos.png}
\caption{Due sensori moderni. A sinistra un sensore CMOS a destra uno CCD} 
\end{figure}
Il CMOS è un sensore digitale che offre una buona qualità di immagine ad costo minore;
per contro l'immagine presenta un forte rumore a causa della conversione 
analogico-digitale.\\
La visione artificiale ha principalmente tre utilizzi:
\begin{itemize}
\item \textbf{Ricognizione:} ricercare uno o più oggetti, scelti a priori, e organizzarli in 
insiemi generici o classi mantenendo informazioni riguardo il loro posizionamento 
nella scena. \\Esempio: ricercare in un'immagine o in un video tutte le persone, 
le macchine o gli animali. 
\item \textbf{Identificazione:} identificare una istanza specifica di una classe 
di oggetti. \\Esempio: ricercare in un'immagine o in un video un volto, 
una macchina o un animale specifico.. 
\item \textbf{Individuazione:} cercare una condizione specifica nell'immagine. 
\\Esempio: cercare imperfezioni nelle immagini a raggi X di superfici o materiali.
\end{itemize}
Un campo che ne fa massiccio utilizzo è la modellazione di ambienti 3D a partire 
da due o più immagini 2D; questo è stato reso possibile 
grazie all'enorme aumento della capacità di elaborazione grafica delle GPU.\\
La visione artificiale è stata applicata a molti altri campi nei quali si è verificata 
una vera e propria rivoluzione.
Uno di questi è la medicina nella quale l'introduzione di questa tecnica ha portato
alla realizzazione di radiografie, angiografie,
tomografie e molte altre; in questo modo è possibile identificare anomalie e 
problemi, quali i tumori, che non sarebbero visibili all'occhio umano 
se non in seguito a procedure molto invasive.
Un altro campo è quello del controllo di veicoli autonomi i quali stanno aumentando
ad un ritmo vertiginoso, progressivamente al migliorarsi delle tecniche di 
visione artificiale. Autovetture, droni, robot e carri per il rifornimento 
sono solo degli esempi a cosa può portare la visione artificiale applicata nei settori 
civili e di ricerca.
\begin{figure}[!htb] \center
\includegraphics[scale=0.2]{immagini/amazon-air.png}
\caption{Drone della società statunitense Amazon. Consegnerà prodotti a domicilio in completa autonomia} 
\end{figure}
La nostra tesi si concentra proprio sull'applicazione ad un robot autonomo dei 
concetti di visione artificiale appena descritti.

