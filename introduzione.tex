%!TEX root = Tesi__Simone_Mariotti.tex
\chapter*{Introduzione}
\fancyhead[R]{\bfseries Introduzione} 	
\fancyfoot[C]{\thepage } 
\addcontentsline{toc}{chapter}{Introduzione}
I primi automi eseguivano azioni preimpostate semplici e ripetitive; erano da considerarsi poco 
più che macchine utensili. Si pensi ai robot delle catene di produzione di un'autovettura, 
eseguono sempre le stesse azioni, si muovono sempre degli stessi centimetri o millimetri 
portandosi sempre nella stessa posizione, eseguono la loro azione programmata 
e passano al successivo compito. Gli algoritmi che governano tali robot, data la 
loro natura e le specifiche che rispettano, non possono essere considerati intelligenti.\\
Intelligente, in ambito informatico, ha una definizione completamente differente 
dall'intelligenza così come la percepiamo in ambito umano. Una macchina non 
possiede creatività, non impara e non immagina. Se sembra possedere una di queste 
qualità è perché è stata programmata in tal senso. Una macchina non può andare oltre 
la sua programmazione, o almeno non ha ancora potuto. Nessuna macchina ad oggi ha superato 
il Test di Turing\footnote{È una condizione proposta da Alan Turing nel 1950 
che una macchina deve soddisfare per essere considerata intelligente.}. Eugene 
Goostman, un software appositamente sviluppato per emulare il pensiero 
di un ragazzo di 13 anni di origine Ucraina e con una limitata conoscenza dell'inglese,
è riuscito ad ``ingannare'' solo il 29\% dei giudici evitando le domande in modo 
credibile per un ragazzo con le caratteristiche dichiarate. Joshua Tenenbaum,
Professore Ph.D. di Scienze Cognitive Computazionali al MIT, in merito ha dichiarato che 
il risultato non è impressionante.\cite{eugene}\\
In informatica un algoritmo è considerato intelligente
se a seguito dell'analisi di un problema sceglie la soluzione più giusta o più efficiente 
sulla base delle informazioni che possiedono.

In questo lavoro di tesi la nostra intelligenza artificiale realizza una forma semplificata del \emph{bin-picking},
tecnica usata in ambito industriale per identificare e prelevare oggetti 
disposti alla rinfusa in un ambiente.\cite{bin-picking} Nel nostro caso si limiterà a due compiti 
``classici'' per le IA\footnote{Intelligenza Artificiale.}, cercare e identificare 
il suo obiettivo. All'avvio del sistema l'app Android che svolge le funzioni di IA,
ci chiederà di selezionare tramite touch screen un colore visibile sul video 
proveniente dalla camera. Una volta selezionato il colore e premuto il pulsante start, 
il robot si muoverà autonomamente alla ricerca di tale colore nell'ambiente. 
Date le caratteristiche dell'algoritmo implementato, l'ambiente di test potrà 
avere qualsiasi forma ed estensione; per stabilire i suoi confini potremo utilizzare 
un materiale a bassa riflessività, come carta nera opaca, oppure un oggetto per 
formare un piano sopraelevato rispetto all'area circostante.
Il robot si aggirerà per l'ambiente evitando ostacoli e senza 
uscire dall'area delimitata finché non troverà un oggetto del colore cercato. 
A questo punto avvertirà di aver raggiunto l'obiettivo e si metterà in attesa di un nuovo 
input da parte dell'utente.