%!TEX root = Tesi__Simone_Mariotti.tex
\chapter*{Conclusioni}
\addcontentsline{toc}{chapter}{Conclusioni}
\fancyfoot[C]{\thepage}
Il robot riesce 
a distinguere colori che siano distanti tra loro 10$^\circ$ nel modello di colori HSV
il che permette di cercare circa 36 colori differenti. Data l'assenza di conoscenza 
dell'ambiente circostante il robot non segue un percorso efficiente per la ricerca ma trova comunque l'obiettivo
nel 95\% dei casi. Molte situazioni di stallo si risolvono grazie alla randomizzazione 
di alcune parti del processo decisionale. Il 5\% di fallimenti è dovuto a situazioni di stallo 
che si verificano in situazioni in cui, vicini ai bordi dell'ambiente, sono posizionati 
due o più ostacoli. Il 95\% di successi supera ampiamente il limite posto in fase di progettazione, 
quindi tutti gli obiettivi prefissati per questo lavoro di tesi sono stati raggiunti in modo
ottimale.
La grande modularità del software ha di fatto creato un semplice framework per 
la creazione di nuove intelligenze artificiali da usare con una qualsiasi piattaforma
composta da un'app Android ed un Arduino DUE. Come visto nel Capitolo 4, l'estensione della
classe BaseAI e l'implementazione del metodo \emph{think()} è tutto quello che serve per
creare un algoritmo di IA funzionante.
\subsection* {Sviluppi futuri}
Il software è stato rilasciato con licenza Open Source è quindi migliorabile, 
modificabile e personalizzabile da futuri studenti o semplici appassionati.
Miglioramenti possibili sono:
\begin{itemize}
\item aggiungere sensori per la navigazione come GPS, 
bussole e giroscopi. Permetterebbero di eliminare la navigazione randomizzata e seguire invece un percorso efficiente ed efficace
di ricerca dell'obiettivo.
\item mappare gli ostacoli trovati in modo da ottenere una mappa dell'ambiente di test.
\item mappare gli oggetti di colori diversi trovati in modo da poterli più facilmente localizzare 
in una successiva ricerca.
\item aggiungere un dispositivo prensile in modo da poter manipolare l'obiettivo una volta trovato.
\end{itemize}

